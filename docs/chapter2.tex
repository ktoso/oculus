\chapter{Architecture overview}
In this chapter I will present a broad overview on the system's design and various components.
It should also provide some background to why a distributed file system was required in order to enable this system,
and why only a distributed system is suited to handle the kind of jobs that Oculus is designed for.

\section{Hadoop Distributed File System - HDFS}

The primary enabeler for this project, and many ,,big data'' projects in the recent years is Hadoop and it's Distributed File System -- for short \textit{HDFS}.

Hadoop is a Java implementation of the ,,Map Reduce'' white paper published in 2004 by Google \todo{need ref}. It first  developed internally at \textit{Yahoo} and then open-sourced in 2007 and is now under the Apache Software Foundation's umbrella, 

\todo{TODO explain why HDFS is the reasonable choice here}

\section{Distributed Actor System - Akka}

In order to enable fast downloading and uploading of video material into the system, an Actor System based solution was implemented.
The used framework, which provided the basic Actor building blocks, is Akka, a JVM based implementation of the Erlang Actor Model of Concurrency. The implementation language selected was Scala, as it's a first-class citizen in the Akka world, as well as for it's conciseness and semantic power.

\todo{image for actor system here}

The Actor Model of computation raises the level of abstraction we operate on in parallel applications beyond threads, and instead
we use Actors. An Actor can be described as an entity that can only recieve and send messages. An external system (the so-called "Actor System") is responsible for assigning Threads to Actors, so that they can perform the work - upon recieving a message.

This model has the advantage of hiding any state the Actor may have from other Actors it interacts with, thus there are by definition no race conditions as no shared-mutable state can be observed in such a system. While the Actor Model of concurrency was designed for local interactions - that is, avoiding shared state in highly parallel systems it also has a significant impact on distributed systems. Because actors share no state, and it is only possible to interact with an Actor using messages it does not matter \textit{where} an Actor is residing -- for example it might reside on another node in our computation cluster. This property was used in Oclus in order to distribute the work-load related to downloading, pre-processing as well as initially uploading the raw video content into the distributed file system.


\todo{TODO why do we need actors here? why does it make sense to distribute the work}